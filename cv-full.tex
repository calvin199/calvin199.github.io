\documentclass{article}
\usepackage{hyperref}
\usepackage{array,longtable}
\usepackage{helvet}
%\usepackage{xcolor}
\usepackage[bmargin=0.85in,tmargin=0.85in,lmargin=1.0in,rmargin=1.0in]{geometry}
\usepackage[dvipsnames]{xcolor}
\pagestyle{empty}
\newcommand{\caul}[1]{\textcolor{YellowGreen}{\underline{\textcolor{black}{#1}}}}
\newcommand{\mlul}[1]{\textcolor{RubineRed}{\underline{\textcolor{black}{#1}}}}
\newcommand{\cbul}[1]{\textcolor{Dandelion}{\underline{\textcolor{black}{#1}}}}
\begin{document}
\hspace{0.163\linewidth}{\bfseries\Large Calvin McCarter}
\hfill \url{http://calvinmccarter.com/}
\vspace{-5pt}

\hrulefill
  \begin{longtable}{@{}>{\raggedleft}p{0.17\linewidth}
     p{\dimexpr0.83\linewidth-2\tabcolsep-\arrayrulewidth\relax}@{}}
 \textsc{About Me} 
     & AI researcher and engineer, with
       \hfill Boston, MA \\
 \textsc{ } 
     & $\triangleright$ first-author papers in NeurIPS and AISTATS,
       \hfill \href{mailto:mccarter.calvin@gmail.com}{mccarter.calvin@gmail.com} \\
 \textsc{ } 
     & $\triangleright$ extensive open-source software contributions, and
       \hfill (616) 272-0909 \\
     & $\triangleright$ experience training neural networks at fast-paced teams. \\[1.5em]
 \textsc{Education} 
     & \textbf{Carnegie Mellon University}, 
       Pittsburgh, PA \\[0.25em]
     & \textit{Ph.D. in Machine Learning}
       \hfill\textbf{August 2013 - May 2019} \\[0.25em]
     & $\triangleright$ \ Advisor: Seyoung Kim
       \hfill GPA: 3.80 \\[0.25em]
     & $\triangleright$ \ Selected Courses: 
       Probabilistic Graphical Models, Convex Optimization, 
       Statistical Machine Learning, 
       Foundations of Machine Learning Theory \\[2.0em]
       %Hidden Markov Models, Graduate Molecular Biology  \\[2.0em]
     & \textbf{University of Michigan}, Ann Arbor, MI \\[0.25em]
     & \textit{Bachelor of Science in Engineering}
       \hfill\textbf{August 2009 - May 2013} \\[0.25em]
     & $\triangleright$ \ Major: Computer Science, Minor: Mathematics 
       \hfill GPA: 3.98 \\[0.25em]
     & $\triangleright$ \ Selected Courses: 
       Operating Systems, Computer Architecture, Database Systems, 
       Numerical Methods, Linear Algebra, Theoretical Statistics\\[2.5em]
 \textsc{Publications}
  & J Michuda, et int., \cbul{C McCarter}, et int., T Taxter,
    ``Validation of a transcriptome-based assay for classifying cancers of unknown primary origin'',
    \href{https://doi.org/10.1007/s40291-023-00650-5}{\textit{Mol Diagn Ther}}, 2023. \\[1.5em]
 \vspace{-25pt} \\{\color{YellowGreen}{CompArch}} \\ {\color{Dandelion}{CompBio}} \\{\color{RubineRed} ML} 
  & \mlul{C McCarter} and N Dronen,
    ``Look-ups are not (yet) all you need for deep learning inference.''
    \href{https://arxiv.org/abs/2207.05808}{\textit{Sparsity in Neural Networks Workshop}}, 2022. \\[1.5em]
  & R Hanson, D Martin, \cbul{C McCarter}, J Paulson,
    ``If Loud Aliens Explain Human Earliness, Quiet Aliens Are Also Rare.''
    \href{https://doi.org/10.3847/1538-4357/ac2369}{\textit{The Astrophysical Journal (APJ)}}, 2021. \\[1.5em]
  & LE Fernandes, et int., \cbul{C McCarter}, et al., ``Real-world Evidence of Diagnostic Testing and Treatment Patterns
    in US Breast Cancer Patients with Implications for Treatment Biomarkers from RNA-sequencing Data.''
    \href{https://doi.org/10.1016/j.clbc.2020.11.012}{\textit{Clinical Breast Cancer}}, 2020. \\[1.5em]
  & \cbul{C McCarter}, J Howrylak, S Kim,
    ``Learning Gene Networks Underlying Clinical Phenotypes Using SNP Perturbations'',
    \href{https://doi.org/10.1371/journal.pcbi.1007940}{\textit{PLOS Computational Biology}}, 2020. \\[1.5em]
  & \mlul{C McCarter} and S Kim, 
    ``Large-Scale Optimization Algorithms for Sparse Conditional Gaussian Graphical Models'',
    \href{http://proceedings.mlr.press/v51/mccarter16.html}{\textit{AISTATS}}, 2016. \\[1.5em]
  & \mlul{C McCarter} and S Kim, 
    ``On Sparse Gaussian Chain Graph Models'', 
    \href{https://proceedings.neurips.cc/paper/2014/file/81c650caac28cdefce4de5ddc18befa0-Paper.pdf}{\textit{NeurIPS}}, 2014. \\[0.5em]
  & S Moon, \mlul{C McCarter}, YH Kuo,
    ``Active learning with partially featured data'',
    \textit{Proceedings of the 23rd International Conference on World Wide Web (WWW)},
    2014. \\[1.5em]
  & \cbul{C McCarter}, et int., B Haab,
    ``Prediction of Glycan Motifs Using Quantitative Analysis of Multi-lectin Binding'',
    \textit{Proteomics Clinical Applications}, 
    7: 9-10, 2013.\\[1.5em]
  & D Chatterjee, \caul{C McCarter}, V Bertacco, 
    ``Simulation-based Signal Selection for State Restoration in Silicon Debug'',
    \textit{ICCAD}, 
    2011. \\[2.5em]
 \textsc{Preprints}
  & \mlul{C McCarter}, ``Towards Backwards-Compatible Data with Confounded Domain Adaptation'',
    \href{https://arxiv.org/abs/2203.12720}{arXiv:2203.12720}, 2022. \\[1.5em]
  & A Basumallik, D Bunandar, N Dronen, L Levkova, \mlul{C McCarter}, L Nair, D Walter, D Widemann,
    ``Adaptive Block Floating-Point for Analog Deep Learning Hardware'',
    \href{https://arxiv.org/abs/2205.06287}{arXiv:2205.06287}, 2022. \\[1.5em]
  & \\[0em]
 \textsc{Patents}
  & D Bunandar, \caul{C McCarter}, A Basumallik, ``Improving the accuracy of analog linear processor.''
    US Patent App. 18/077,177 (2022).\\[0em] 
  & J Michuda, \dots, \cbul{C McCarter}, et al., ``Systems and methods for multilabel cancer classification.''
    US Patent App. 17/150,992 (2021).\\[0.0em]
  & \\[0em]
\newpage
 \textsc{Experience} 
     & \textbf{Amazon Alexa AI}, Applied Scientist
       \hfill\textbf{June 2022 - Present}\\[0.25em]
     & Researching and developing neural retrieval and ranking methods for entity resolution.
       Training and deploying to newly-onboarding Alexa use-cases.\\[1em]
     &\\[-0.5em]
     & \textbf{Independent Project}
       \hfill\textbf{February 2022 - June 2022}\\[0.25em]
     & Developed domain adaptation method which conditions on confounders,
       and explored founding a startup to ease deployment of vector embeddings across biology.\\[1em]
     &\\[-0.5em]
     & \textbf{Lightmatter}, ML Scientist
       \hfill\textbf{January 2021 - February 2022}\\[0.25em]
     & Researched ways to accelerate deep learning inference on photonic AI accelerator.
       Explored finetuning strategies to ensure model accuracy despite hardware noise 
       and quantization.
       Helped invent next generation of hardware architecture to improve noise robustness.\\[3em]
     &\\[-0.5em]
     & \textbf{Tempus Labs}, ML Scientist
       \hfill\textbf{June 2019 - January 2021}\\[0.25em]
     & Created and validated a new batch effect correction method,
       which was deployed on the Tempus RNA-seq pipeline as 
       the source-of-truth for all clinical AI models and 
       pharma data deliveries. 
       Developed a new topic model for gene expression deconvolution
       in metastatic cancers.
       Explored network learning methods and graph neural nets for
       gene expression networks and chromosomal rearrangement graphs.\\[5em]
     &\\[-0.5em]
     & \textbf{Carnegie Mellon University}, PhD Student
       \hfill\textbf{August 2013 - May 2019} \\[0.25em]
     & Developed novel sparse graphical models 
       and scalable optimization algorithms for disease systems biology.
       Used statistical learning to discover the gene regulatory networks 
       which explain the effect of genetic variation on clinical traits.\\[3em]
     & \textbf{Van Andel Research Institute}, Research Intern
       \hfill\textbf{Summer 2013} \\[0.25em]
     & Worked under the supervision of Brian Haab to apply
       feature selection method in pancreatic cancer biomarker discovery
       and to validate method on proteomics database.\\[2em]
     & \textbf{Google}, Software Engineering Intern
       \hfill\textbf{Summer 2012}\\[0.25em]
     & Worked on server backend for Google Flight Search, developing functionality 
       for international results for live Flight Search queries.\\[2em]
     & \textbf{University of Michigan}, Research Assistant
       \hfill\textbf{2011} \\[0.25em]
     & Worked under the supervision of Valeria Bertacco 
       and Debapriya Chatterjee to develop post-silicon validation method.
       Designed and implemented parallel algorithm in CUDA. \\[2.0em]
\iffalse
     & \textbf{Arbor Networks}, Summer Intern
       \hfill\textbf{Summer 2011}\\[0.25em]
     & Implemented instrumentation in deep packet inspection system and
       prepared performance analysis tools geared to IPv6 transition.\\[2em]
     & \textbf{University of Michigan}, Research Assistant
       \hfill\textbf{Summer 2010} \\[0.25em]
     & Analyzed data from simulated advertising auctions under the 
       supervision of Michael Wellman to understand impact of bidding strategies
       on advertiser profitability.\\[0.0em]
\fi
%\newpage
\iffalse
 \textsc{Awards}
     & Outstanding Research Award, University of Michigan EECS Department, 2013 \\
     & Henry Ford II Prize, University of Michigan College of Engineering, 2012 \\
     & James B. Angell Scholar, University of Michigan, 2012 \\
     & 1st Place, Cooley Essay Writing Contest, University of Michigan, 2011 \\
     & National Merit Scholar, 2009 \\
     & Finalist, US National Chemistry Olympiad, 2009 \\%[1.5em]
     & National Champion, National Geographic Bee, 2002 \\[1.5em]
\fi

 \textsc{Selected Open-Source Contributions}
  & ConDo
    \hfill \url{https://github.com/calvinmccarter/condo-adapter}\\[-2.5em]
  & Toolbox for Confounded Domain Adaptation.
    \hfill [author]\\[0.25em]
  & onnx2pytorch
    \hfill \url{https://github.com/ToriML/onnx2pytorch}\\[0em]
  & Converts ONNX models to PyTorch.
    \hfill [\href{https://github.com/ToriML/onnx2pytorch/graphs/contributors}{main contributor}]\\[0.25em]
  & PerturbNet
    \hfill \url{https://github.com/SeyoungKimLab/PerturbNet}\\[0em]
  & Learns multi-omic gene regulatory networks.
     \hfill [\href{https://github.com/SeyoungKimLab/PerturbNet/graphs/contributors}{author}]\\[0.25em]
  & MLPerf Inference
    \hfill \url{https://github.com/mlcommons/inference}\\[0em]
  & Deep learning benchmark.
    \hfill [\href{https://github.com/mlcommons/inference/pull/1015}{memory-efficient pyramidal encoder for RNN-Transducer}]\\[0.25em]
  & matrix-completion
    \hfill \url{https://github.com/tonyduan/matrix-completion}\\[0em]
  & Classical matrix completion.
    \hfill [\href{https://github.com/tonyduan/matrix-completion/pull/3}{incremental singular-vector thresholding}]\\[0.25em]
  & PyTorch
    \hfill \url{https://github.com/pytorch/pytorch}\\[0em]
  & Deep learning framework.
    \hfill [\href{https://github.com/pytorch/pytorch/commits?author=calvinmccarter-at-lightmatter}{added LazyInstanceNorm}]\\[0.25em]
  & nanopq
    \hfill \url{https://github.com/matsui528/nanopq}\\[0em]
  & Product quantization (PQ) and optimized PQ.
    \hfill [\href{https://github.com/matsui528/nanopq/graphs/contributors}{eigenvalue allocation initialization}]\\[1.0em]
 \textsc{Languages}
  & Python (PyTorch, TensorFlow, pandas, NumPy, Numba), C, C++, CUDA, Matlab\\[1.0em]%, Shell, \LaTeX\\[2.5em]
%\newpage
 \textsc{Presentations}
  & \textit{Transcriptome background tissue correction in metastatic cancers using a correlated composition admixture model.}\\
  & American Association for Cancer Research (AACR), Annual Meeting 2020.\\
  & \textit{An efficient algorithm for learning a gene network underlying clinical phenotypes under SNP perturbations.}\\
  & Genome Informatics meeting at Cold Spring Harbor Labs, November 2017.\\
  & \textit{Penalised Additive Least Squares Models for High Dimensional Nonparametric Regression and Function Selection.}\\
  & \href{http://calvinmccarter.com/papers/additivekernel.pdf}{Large-Scale Kernel Learning Workshop @ ICML} 2015.\\[1.5em]
\iffalse
  & \textit{Multi-modal structure learning in high dimensions for integrative genomics.}\\
  & Machine Learning Lunch Seminar. Carnegie Mellon University, October 2015.\\[1.5em]
\fi
 \textsc{Teaching}
  %& \textit{Correlation, Causation, Confounding, and Conditioning} (Guest Lecturer, UPenn)
  %  \hfill Spring 2022\\
  & \textit{Probabilistic Graphical Models} (Teaching Assistant, CMU)
    \hfill \textbf{Spring 2016}\\
  & \textit{Introduction to Machine Learning} (Teaching Assistant, CMU)
    \hfill \textbf{Fall 2015}\\[1.5em]
 \textsc{Activities and Professional Service}
  & \textit{Elicitation of latent knowledge (ELK) award contest}
  \hfill\textbf{February 2022}\\[-2.5em]
  & AI Alignment Research Center (\href{https://calvinmccarter.wordpress.com/2022/02/19/mind-blindness-strategy-for-eliciting-latent-knowledge/}{research proposal}
      received \href{https://www.alignmentforum.org/posts/zjMKpSB2Xccn9qi5t/elk-prize-results}{honorable mention}).\\[0.5em]
  & \textit{Paper reviewing}
  \hfill\textbf{June 2016 - Present}\\
  & Reviewer for \textit{NeurIPS}, \textit{AISTATS}, \textit{IEEE Internet of Things}, and \textit{SciPy}.\\[0.5em]
  & \textit{University of Pittsburgh Biomedical Informatics Training Program}
    \hfill\textbf{2017}\\
  & Mentor to undergraduate research intern through iBRIC program.\\[0.5em]
  & \textit{Middle school science fair judging}
    \hfill\textbf{2015-2020}\\
  & Science fair judge for PA Junior Academy of Science and Chicago Public Schools.\\[0.5em]
  %& \textit{Machine Learning Department Admissions Committee}
  %  \hfill\textbf{2015}\\
  %& Reviewed application materials of prospective graduate students.\\[0.5em]
  & \textit{Machine Learning Department Student Research Symposium}
    \hfill\textbf{2014}\\
  & Member of organizing committee. Created website and helped plan symposium.\\[0.5em]
  %& \textit{CMU Language Technologies Institute Research Colloquium}
  %  \hfill\textbf{2013 - 2014}\\ 
  %& Helped organize weekly research seminar as member of student planning committee.\\[0.5em]
  & \textit{English Language Institute Conversation Circle Program}
    \hfill\textbf{2011 - 2013}\\
  & Group leader of conversation circle for ESL students at University of Michigan.\\[0.5em]
  & \textit{University of Michigan Robocup (Robot Soccer) Team}
    \hfill\textbf{2009 - 2012}\\
  & Member and team leader (2010-2011). Developed computer vision subsystem.\\[1.5em]
% add Tempus causal inference TBOE
% add science fair judging
% add open source contribs
\end{longtable}
\end{document}
